\documentclass[]{elsarticle} %review=doublespace preprint=single 5p=2 column
%%% Begin My package additions %%%%%%%%%%%%%%%%%%%
\usepackage[hyphens]{url}
\usepackage{lineno} % add
\providecommand{\tightlist}{%
  \setlength{\itemsep}{0pt}\setlength{\parskip}{0pt}}

\bibliographystyle{elsarticle-harv}
\biboptions{sort&compress} % For natbib
\usepackage{graphicx}
\usepackage{booktabs} % book-quality tables
%% Redefines the elsarticle footer
%\makeatletter
%\def\ps@pprintTitle{%
% \let\@oddhead\@empty
% \let\@evenhead\@empty
% \def\@oddfoot{\it \hfill\today}%
% \let\@evenfoot\@oddfoot}
%\makeatother

% A modified page layout
\textwidth 6.75in
\oddsidemargin -0.15in
\evensidemargin -0.15in
\textheight 9in
\topmargin -0.5in
%%%%%%%%%%%%%%%% end my additions to header

\usepackage[T1]{fontenc}
\usepackage{lmodern}
\usepackage{amssymb,amsmath}
\usepackage{ifxetex,ifluatex}
\usepackage{fixltx2e} % provides \textsubscript
% use upquote if available, for straight quotes in verbatim environments
\IfFileExists{upquote.sty}{\usepackage{upquote}}{}
\ifnum 0\ifxetex 1\fi\ifluatex 1\fi=0 % if pdftex
  \usepackage[utf8]{inputenc}
\else % if luatex or xelatex
  \usepackage{fontspec}
  \ifxetex
    \usepackage{xltxtra,xunicode}
  \fi
  \defaultfontfeatures{Mapping=tex-text,Scale=MatchLowercase}
  \newcommand{\euro}{€}
\fi
% use microtype if available
\IfFileExists{microtype.sty}{\usepackage{microtype}}{}
\ifxetex
  \usepackage[setpagesize=false, % page size defined by xetex
              unicode=false, % unicode breaks when used with xetex
              xetex]{hyperref}
\else
  \usepackage[unicode=true]{hyperref}
\fi
\hypersetup{breaklinks=true,
            bookmarks=true,
            pdfauthor={},
            pdftitle={Short Paper},
            colorlinks=true,
            urlcolor=blue,
            linkcolor=magenta,
            pdfborder={0 0 0}}
\urlstyle{same}  % don't use monospace font for urls
\setlength{\parindent}{0pt}
\setlength{\parskip}{6pt plus 2pt minus 1pt}
\setlength{\emergencystretch}{3em}  % prevent overfull lines
\setcounter{secnumdepth}{0}
% Pandoc toggle for numbering sections (defaults to be off)
\setcounter{secnumdepth}{0}
% Pandoc header


\usepackage[nomarkers]{endfloat}

\begin{document}
\begin{frontmatter}

  \title{Short Paper}
    \author[Some Institute of Technology]{Alice Anonymous\corref{c1}}
   \ead{alice@example.com} 
   \cortext[c1]{Corresponding Author}
    \author[Another University]{Bob Security}
   \ead{bob@example.com} 
  
      \address[Some Institute of Technology]{Department, Street, City, State, Zip}
    \address[Another University]{Department, Street, City, State, Zip}
  
  \begin{abstract}
  This is the abstract.
  
  It consists of two paragraphs.
  \end{abstract}
  
 \end{frontmatter}

\emph{Text based on elsarticle sample manuscript, see
\url{http://www.elsevier.com/author-schemas/latex-instructions\#elsarticle}}

\hypertarget{the-elsevier-article-class}{%
\section{The Elsevier article class}\label{the-elsevier-article-class}}

\hypertarget{introduction}{%
\section{Introduction}\label{introduction}}

Intracerebral hemorrhage (ICH) is a neurological condition that results
from a blood vessel rupturing into the tissue and possibly extending
into the ventricles of the brain. The use of X-ray computed tomography
(CT) scans allows clinicians and researchers to qualitatively and
quantitatively describe the characteristics of a hemorrhage to guide
interventions and treatments. CT scanning is widely available and is the
most commonly used diagnostic tool in patients with ICH (Sahni and
Weinberger 2007). The volume of ICH has been consistently demonstrated
to be an important diagnostic predictor of stroke severity, long-term
functional outcome, and mortality (Broderick et al. 1993; Hemphill et
al. 2001; Tuhrim et al. 1999). ICH volume change is also a common
primary outcome (Anderson et al. 2008, 2010; Qureshi et al. 2011; Mayer
et al. 2005) and secondary outcome (T. Morgan et al. 2008; Anderson et
al. 2008; T Morgan et al. 2008) in clinical trials. Moreover, the
location of the ICH has been shown to affect functional outcome in
patients with stroke (Rost et al. 2008; Castellanos et al. 2005). Thus,
quantitative measures of ICH (e.g.\textasciitilde{}volume, location, and
shape) are increasingly important for treatment and other clinical
decision.

ICH volume can be estimated quickly, for example, using the ABC/2 method
(Broderick et al. 1993). In this method, a reader chooses the slice with
the largest area of hemorrhage. The length of the intersection between
this first axis and the hemorrhage is denoted by A. The next step is to
draw an orthogonal line at the middle of the segment of length A in the
same plane that contains the largest hemorrhage area. The length of the
intersection between this second orthogonal axis and the hemorrhage is
denoted by B. The reader then counts the number of slices where
hemorrhage is present (C). The volume estimate is
\(\frac{A\times B\times C}{2}\), which is an approximation of the volume
under the assumption that the hemorrhage shape is well approximated by
an ellipsoid (Kothari et al. 1996). As this method is relatively easy to
implement in practice, it can be used to quickly produce rough estimates
of hemorrhage volume (Webb et al. 2015).

Although ABC/2 is widely used; Divani et al. (2011) found that the
measurement error associated with the ABC/2 method were significantly
greater than those using planimetry, which requires slice-by-slice
hemorrhage segmentation by trained readers. Planimetry is much more
labor intensive and time consuming, but it more accurately estimates the
true ICH volume compared to the ABC/2 approach, especially for
irregularly shaped ICH and for smaller thickness
(i.e.\textasciitilde{}higher resolution) scans.

Another problem that has not been discussed in the literature is that
ICH may change over time. The shape of the ICH may initially be well
approximated by an ellipsoid but the approximation may become
increasingly inaccurate over time as the lesion changes shape, migrates
through the surrounding tissues, or breaks down. Surgical interventions
that target the removal of ICH may also change the shape of the ICH or
cause additional bleeding.

Moreover, the ABC/2 method has been shown to consistently over-estimate
infarct volume (Pedraza et al. 2012) and may have significant
inter-rater variability {[}Hussein et al. (2013)\}. Therefore, a rapid,
automated, and validated method for estimating hemorrhage location and
its volume from CT scans is highly relevant in clinical trials and
clinical care. Accuracy is accompanied by increase of both diagnostic
and prognostic value.

Methods have been proposed for segmentation of ICH using magnetic
resonance images (MRI) (Wang et al. 2013; Carhuapoma et al. 2003).
However, in most clinical settings CT, not MRI, is the image of choice.
Furthermore, MRI sequences and protocols may vary across sites and there
is no general, standardized, agreed-upon MRI protocol for ICH
standard-of-care. Thus, there is a need for ICH segmentation that relies
only on CT scan information, is reliable, reproducible, available, and
well validated against planimetry.

We propose an algorithm that can estimate the probability of ICH at the
voxel level, produce a binary image of ICH location, and estimate ICH
volume. We will compare our predicted ICH maps to the gold standard --
manual segmentation. Several methods have been presented for automated
methods for estimating ICH from CT scans (Prakash et al. 2012; Loncaric,
Cosic, and Dhawan 1996; Loncaric et al. 1999; Pérez et al. 2007;
Gillebert, Humphreys, and Mantini 2014). These methods include fuzzy
clustering (Prakash et al. 2012; Loncaric, Cosic, and Dhawan 1996),
simulated annealing (Loncaric et al. 1999), 3-dimensional (3D)
mathematical morphology operations (Pérez et al. 2007), and
template-based comparisons (Gillebert, Humphreys, and Mantini 2014).
Unfortunately, no software for ICH segmentation is publicly available.

We provide a completely automated pipeline of analysis from raw images
to binary hemorrhage masks and volume estimates, and provide a public
webpage to test the software.

\hypertarget{introduction-1}{%
\section{Introduction}\label{introduction-1}}

\hypertarget{data}{%
\subsection{Data}\label{data}}

\hypertarget{participants-and-imaging-data}{%
\subsection{Participants and Imaging
Data}\label{participants-and-imaging-data}}

We used CT images from patients enrolled in the MISTIE II (Minimally
Invasive Surgery plus recombinant-tissue plasminogen activator for
Intracerebral Hemorrhage Evacuation) stroke trial (T. Morgan et al.
2008). We analyzed \(641\) scans taken prior to randomization and
treatment, corresponding to the first scan acquired post-stroke for
\(641\) unique patients. Inclusion criteria into the study included:
\(18\) to \(80\) years of age and spontaneous supratentorial
intracerebral hemorrhage above \(20\) milliliters (mL) in size (for full
criteria, see Mould et al. (2013)). The population analyzed here had a
mean (standard deviation (SD)) age of \(59.1123245\) \((11.3240557)\)
years, was \(58\%\) male, and was 59.8, 33.1, 3.7, 2.2, 0.9, 0.2, 0.2.
CT data were collected as part of the Johns Hopkins Medicine
IRB-approved MISTIE research studies with written consent from
participants.

The study protocol was executed with minor, but important, differences
across the \(1\) sites.\\
 In head CT scanning, the gantry may be tilted for multiple purposes,
for example, so that sensitive organs, such as the eyes, are not exposed
to X-ray radiation. This causes scan slices to be acquired at an oblique
angle with respect to the patient. Gantry tilt was observed in \(0\)
scans.

Slice thickness of the image varied within the scan for 0 scans.

For example, a scan may have \(10\) millimeter (mm) slices at the top
and bottom of the brain and \(5\)mm slices in the middle of the brain.
Therefore, the original scans analyzed had different voxel (volume
element) dimensions. These conditions are characteristic of how scan are
presented in many diagnostic cases.

\hypertarget{hemorrhage-segmentation-and-location-identification}{%
\subsection{Hemorrhage Segmentation and Location
Identification}\label{hemorrhage-segmentation-and-location-identification}}

ICH was manually segmented on CT scans using the OsiriX imaging software
by expert readers (OsiriX v. 4.1, Pixmeo; Geneva, Switzerland). After
image quality review, continuous, non-overlapping slices of the entire
hemorrhage were segmented. Readers employed a semiautomated
threshold-based approach using a Hounsfield unit (HU) range of \(40\) to
\(80\) to select potential regions of hemorrhage (Bergström et al. 1977,
smith\_imaging\_2006); these regions were then further quality
controlled and refined by readers using direct inspection of images.
Binary hemorrhage masks were created by setting voxel intensity to \(1\)
if the voxel was classified as hemorrhage, regardless of location, and
\(0\) otherwise.

\hypertarget{image-processing-brain-extraction-registration}{%
\subsection{Image Processing: Brain Extraction,
Registration}\label{image-processing-brain-extraction-registration}}

CT images and binary hemorrhage masks were exported from OsiriX to DICOM
(Digital Imaging and Communications in Medicine) format. The image
processing pipeline can be seen in
Figure\textasciitilde{}\ref{fig:framework}. Images with gantry tilt were
corrected using a customized MATLAB (The Mathworks, Natick,
Massachusetts, USA) user-written script
(\{\scriptsize \url{http://bit.ly/1ltIM8c}\}). Images were converted to
the Neuroimaging Informatics Technology Initiative (NIfTI) data format
using \texttt{dcm2nii} (provided with MRIcro (Rorden and Brett 2000)).
Images were constrained to values \(-1024\) and \(3071\) HU to remove
potential image rescaling errors and artifacts. No interpolation was
done for images with a variable slice thickness. Thickness was
determined from the first converted slice and the NIfTI format assumes
homogeneous thickness throughout the image. In a future release of
\texttt{dcm2nii}, called \texttt{dcm2niix}, interpolation will be done
for scans with variable slice thickness and gantry-tilt correction will
be performed automatically.

All image analysis was done in the R statistical software (R Core Team
2015), using the \emph{fslr} (Muschelli, Sweeney, et al. 2015) package
to call functions from the FSL (Jenkinson et al. 2012) neuroimaging
software (version 5.0.4), and the \emph{ANTsR} package to call functions
from the ANTs (Advanced Normalization Tools) neuroimaging software
(Avants et al. 2011).

Brains were extracted to remove skull, eyes, facial and nasal features,
extracranial skin, and non-human elements of the image captured by the
CT scanner, such as the gantry, pillows, or medical devices. Removal of
these elements was performed using the brain extraction tool (BET)
(Smith 2002), a function of FSL, using a previously published validated
CT-specific brain extraction protocol (Muschelli, Ullman, et al. 2015).

\hypertarget{front-matter}{%
\section{Front matter}\label{front-matter}}

The author names and affiliations could be formatted in two ways:

\begin{enumerate}
\def\labelenumi{(\arabic{enumi})}
\item
  Group the authors per affiliation.
\item
  Use footnotes to indicate the affiliations.
\end{enumerate}

See the front matter of this document for examples. You are recommended
to conform your choice to the journal you are submitting to.

\hypertarget{bibliography-styles}{%
\section{Bibliography styles}\label{bibliography-styles}}

There are various bibliography styles available. You can select the
style of your choice in the preamble of this document. These styles are
Elsevier styles based on standard styles like Harvard and Vancouver.
Please use BibTeX~to generate your bibliography and include DOIs
whenever available.

\hypertarget{references}{%
\section*{References}\label{references}}
\addcontentsline{toc}{section}{References}

\hypertarget{refs}{}
\leavevmode\hypertarget{ref-anderson_effects_2010}{}%
Anderson, Craig S., Yining Huang, Hisatomi Arima, Emma Heeley, Christian
Skulina, Mark W. Parsons, Bin Peng, et al. 2010. ``Effects of Early
Intensive Blood Pressure-Lowering Treatment on the Growth of Hematoma
and Perihematomal Edema in Acute Intracerebral Hemorrhage the Intensive
Blood Pressure Reduction in Acute Cerebral Haemorrhage Trial
(INTERACT).'' \emph{Stroke} 41 (2):307--12.
\url{https://doi.org/10.1161/STROKEAHA.109.561795}.

\leavevmode\hypertarget{ref-anderson_intensive_2008}{}%
Anderson, Craig S., Yining Huang, Ji Guang Wang, Hisatomi Arima, Bruce
Neal, Bin Peng, Emma Heeley, et al. 2008. ``Intensive Blood Pressure
Reduction in Acute Cerebral Haemorrhage Trial (INTERACT): A Randomised
Pilot Trial.'' \emph{The Lancet Neurology} 7 (5):391--99.
\url{http://www.sciencedirect.com/science/article/pii/S1474442208700693}.

\leavevmode\hypertarget{ref-avants_reproducible_2011}{}%
Avants, Brian B., Nicholas J. Tustison, Gang Song, Philip A. Cook, Arno
Klein, and James C. Gee. 2011. ``A Reproducible Evaluation of ANTs
Similarity Metric Performance in Brain Image Registration.''
\emph{NeuroImage} 54 (3):2033--44.
\url{https://doi.org/10.1016/j.neuroimage.2010.09.025}.

\leavevmode\hypertarget{ref-bergstrom_variation_1977}{}%
Bergström, M, K Ericson, B Levander, P Svendsen, and S Larsson. 1977.
``Variation with Time of the Attenuation Values of Intracranial
Hematomas.'' \emph{Journal of Computer Assisted Tomography} 1
(1):57--63.

\leavevmode\hypertarget{ref-broderick_volume_1993}{}%
Broderick, J. P., T. G. Brott, J. E. Duldner, T. Tomsick, and G. Huster.
1993. ``Volume of Intracerebral Hemorrhage. A Powerful and Easy-to-Use
Predictor of 30-Day Mortality.'' \emph{Stroke} 24 (7):987--93.
\url{https://doi.org/10.1161/01.STR.24.7.987}.

\leavevmode\hypertarget{ref-carhuapoma2003brain}{}%
Carhuapoma, Ricardo J, Daniel F Hanley, Mousumi Banerjee, and Norman J
Beauchamp. 2003. ``Brain Edema After Human Cerebral Hemorrhage a
Magnetic Resonance Imaging Volumetric Analysis.'' \emph{Journal of
Neurosurgical Anesthesiology} 15 (3). LWW:230--33.

\leavevmode\hypertarget{ref-castellanos_predictors_2005}{}%
Castellanos, M., R. Leira, J. Tejada, A. Gil-Peralta, A. Dàvalos, and J.
Castillo. 2005. ``Predictors of Good Outcome in Medium to Large
Spontaneous Supratentorial Intracerebral Haemorrhages.'' \emph{Journal
of Neurology, Neurosurgery \& Psychiatry} 76 (5):691--95.
\url{https://doi.org/10.1136/jnnp.2004.044347}.

\leavevmode\hypertarget{ref-divani_abcs_2011}{}%
Divani, Afshin A., Shahram Majidi, Xianghua Luo, Fotis G. Souslian, Jie
Zhang, Aviva Abosch, and Ramachandra P. Tummala. 2011. ``The ABCs of
Accurate Volumetric Measurement of Cerebral Hematoma.'' \emph{Stroke} 42
(6):1569--74.
\url{https://stroke.ahajournals.org/content/42/6/1569.full}.

\leavevmode\hypertarget{ref-gillebert_automated_2014}{}%
Gillebert, Céline R., Glyn W. Humphreys, and Dante Mantini. 2014.
``Automated Delineation of Stroke Lesions Using Brain CT Images.''
\emph{NeuroImage: Clinical} 4:540--48.
\url{https://doi.org/10.1016/j.nicl.2014.03.009}.

\leavevmode\hypertarget{ref-hemphill_ich_2001}{}%
Hemphill, J. Claude, David C. Bonovich, Lavrentios Besmertis, Geoffrey
T. Manley, and S. Claiborne Johnston. 2001. ``The ICH Score a Simple,
Reliable Grading Scale for Intracerebral Hemorrhage.'' \emph{Stroke} 32
(4):891--97. \url{https://doi.org/10.1161/01.STR.32.4.891}.

\leavevmode\hypertarget{ref-hussein_reliability_2013}{}%
Hussein, Haitham M., Nauman A. Tariq, Yuko Y. Palesch, and Adnan I.
Qureshi. 2013. ``Reliability of Hematoma Volume Measurement at Local
Sites in a Multicenter Acute Intracerebral Hemorrhage Clinical Trial.''
\emph{Stroke} 44 (1):237--39.
\url{https://doi.org/10.1161/STROKEAHA.112.667220}.

\leavevmode\hypertarget{ref-jenkinson_fsl_2012}{}%
Jenkinson, Mark, Christian F. Beckmann, Timothy E. J. Behrens, Mark W.
Woolrich, and Stephen M. Smith. 2012. ``FSL.'' \emph{NeuroImage} 62
(2):782--90. \url{https://doi.org/10.1016/j.neuroimage.2011.09.015}.

\leavevmode\hypertarget{ref-kothari_abcs_1996}{}%
Kothari, Rashmi U., Thomas Brott, Joseph P. Broderick, William G.
Barsan, Laura R. Sauerbeck, Mario Zuccarello, and Jane Khoury. 1996.
``The ABCs of Measuring Intracerebral Hemorrhage Volumes.''
\emph{Stroke} 27 (8):1304--5.
\url{https://doi.org/10.1161/01.STR.27.8.1304}.

\leavevmode\hypertarget{ref-loncaric_hierarchical_1996}{}%
Loncaric, Sven, Dubravko Cosic, and Atam P. Dhawan. 1996. ``Hierarchical
Segmentation of CT Head Images.'' \emph{Proc IEEE EMBS. Doi} 10:1109.
\url{http://www.researchgate.net/publication/2690468_Hierarchical_Segmentation_of_CT_Head_Images/file/d912f50edddd1ca672.pdf}.

\leavevmode\hypertarget{ref-loncaric_quantitative_1999}{}%
Loncaric, Sven, Atam P. Dhawan, Dubravko Cosic, Domagoj Kovacević,
Joseph Broderick, and Thomas Brott. 1999. ``Quantitative Intracerebral
Brain Hemorrhage Analysis.'' In \emph{Medical Imaging'99}, 886--94.
International Society for Optics; Photonics.
\url{http://proceedings.spiedigitallibrary.org/proceeding.aspx?articleid=980754}.

\leavevmode\hypertarget{ref-mayer_recombinant_2005}{}%
Mayer, Stephan A., Nikolai C. Brun, Kamilla Begtrup, Joseph Broderick,
Stephen Davis, Michael N. Diringer, Brett E. Skolnick, and Thorsten
Steiner. 2005. ``Recombinant Activated Factor VII for Acute
Intracerebral Hemorrhage.'' \emph{New England Journal of Medicine} 352
(8):777--85. \url{https://doi.org/10.1056/NEJMoa042991}.

\leavevmode\hypertarget{ref-morgan_preliminary_2008_clear}{}%
Morgan, T, I Awad, P Keyl, K Lane, and D Hanley. 2008. ``Preliminary
Report of the Clot Lysis Evaluating Accelerated Resolution of
Intraventricular Hemorrhage (Clear-Ivh) Clinical Trial.'' In
\emph{Cerebral Hemorrhage}, 217--20. Springer.

\leavevmode\hypertarget{ref-morgan_preliminary_2008_mistie}{}%
Morgan, T., M. Zuccarello, R. Narayan, P. Keyl, K. Lane, and D. Hanley.
2008. ``Preliminary Findings of the Minimally-Invasive Surgery Plus rtPA
for Intracerebral Hemorrhage Evacuation (MISTIE) Clinical Trial.'' In
\emph{Cerebral Hemorrhage}, 147--51. Springer.
\url{http://link.springer.com/chapter/10.1007/978-3-211-09469-3_30}.

\leavevmode\hypertarget{ref-mould_minimally_2013}{}%
Mould, W. Andrew, J. Ricardo Carhuapoma, John Muschelli, Karen Lane,
Timothy C. Morgan, Nichol A. McBee, Amanda J. Bistran-Hall, et al. 2013.
``Minimally Invasive Surgery Plus Recombinant Tissue-Type Plasminogen
Activator for Intracerebral Hemorrhage Evacuation Decreases
Perihematomal Edema.'' \emph{Stroke} 44 (3):627--34.
\url{https://doi.org/10.1161/STROKEAHA.111.000411}.

\leavevmode\hypertarget{ref-muschelli2015fslr}{}%
Muschelli, John, Elizabeth Sweeney, Martin Lindquist, and Ciprian
Crainiceanu. 2015. ``fslr: Connecting the FSL Software with R.'' \emph{R
Journal} 7 (1). R Foundation for Statistical Computing:163--75.

\leavevmode\hypertarget{ref-muschelli_validated_2015}{}%
Muschelli, John, Natalie L. Ullman, W. Andrew Mould, Paul Vespa, Daniel
F. Hanley, and Ciprian M. Crainiceanu. 2015. ``Validated Automatic Brain
Extraction of Head CT Images.'' \emph{NeuroImage}.
\url{https://doi.org/10.1016/j.neuroimage.2015.03.074}.

\leavevmode\hypertarget{ref-pedraza_reliability_2012}{}%
Pedraza, Salvador, Josep Puig, Gerard Blasco, Josep Daunis-i-Estadella,
Imma Boada, Anton Bardera, Mar Castellanos, and Joaquín Serena. 2012.
``Reliability of the ABC/2 Method in Determining Acute Infarct Volume.''
\emph{Journal of Neuroimaging} 22 (2):155--59.
\url{https://doi.org/10.1111/j.1552-6569.2011.00588.x}.

\leavevmode\hypertarget{ref-perez_set_2007}{}%
Pérez, Noel, José A. Valdés, Miguel A. Guevara, Luis A. Rodríguez, and
J. M. Molina. 2007. ``Set of Methods for Spontaneous ICH Segmentation
and Tracking from CT Head Images.'' In \emph{Progress in Pattern
Recognition, Image Analysis and Applications}, 212--20. Springer.
\url{http://link.springer.com/chapter/10.1007/978-3-540-76725-1_23}.

\leavevmode\hypertarget{ref-prakash_segmentation_2012}{}%
Prakash, K. N. Bhanu, Shi Zhou, Tim C. Morgan, Daniel F. Hanley, and
Wieslaw L. Nowinski. 2012. ``Segmentation and Quantification of
Intra-Ventricular/Cerebral Hemorrhage in CT Scans by Modified Distance
Regularized Level Set Evolution Technique.'' \emph{International Journal
of Computer Assisted Radiology and Surgery} 7 (5):785--98.
\url{https://doi.org/10.1007/s11548-012-0670-0}.

\leavevmode\hypertarget{ref-qureshi_association_2011}{}%
Qureshi, Adnan I., Yuko Y. Palesch, Renee Martin, Jill Novitzke,
Salvador Cruz-Flores, Asad Ehtisham, Mustapha A. Ezzeddine, et al. 2011.
``Association of Serum Glucose Concentrations During Acute
Hospitalization with Hematoma Expansion, Perihematomal Edema, and Three
Month Outcome Among Patients with Intracerebral Hemorrhage.''
\emph{Neurocritical Care} 15 (3):428--35.
\url{https://doi.org/10.1007/s12028-011-9541-8}.

\leavevmode\hypertarget{ref-RCORE}{}%
R Core Team. 2015. \emph{R: A Language and Environment for Statistical
Computing}. Vienna, Austria: R Foundation for Statistical Computing.
\url{https://www.R-project.org/}.

\leavevmode\hypertarget{ref-rorden_stereotaxic_2000}{}%
Rorden, Chris, and Matthew Brett. 2000. ``Stereotaxic Display of Brain
Lesions.'' \emph{Behavioural Neurology} 12 (4):191--200.
\url{https://doi.org/10.1155/2000/421719}.

\leavevmode\hypertarget{ref-rost_prediction_2008}{}%
Rost, Natalia S., Eric E. Smith, Yuchiao Chang, Ryan W. Snider, Rishi
Chanderraj, Kristin Schwab, Emily FitzMaurice, et al. 2008. ``Prediction
of Functional Outcome in Patients with Primary Intracerebral Hemorrhage
the FUNC Score.'' \emph{Stroke} 39 (8):2304--9.
\url{https://doi.org/10.1161/STROKEAHA.107.512202}.

\leavevmode\hypertarget{ref-sahni_management_2007}{}%
Sahni, Ramandeep, and Jesse Weinberger. 2007. ``Management of
Intracerebral Hemorrhage.'' \emph{Vascular Health and Risk Management} 3
(5):701--9. \url{http://www.ncbi.nlm.nih.gov/pmc/articles/PMC2291314/}.

\leavevmode\hypertarget{ref-smith_fast_2002}{}%
Smith, Stephen M. 2002. ``Fast Robust Automated Brain Extraction.''
\emph{Human Brain Mapping} 17 (3):143--55.
\url{https://doi.org/10.1002/hbm.10062}.

\leavevmode\hypertarget{ref-tuhrim_volume_1999}{}%
Tuhrim, Stanley, Deborah R. Horowitz, Michael Sacher, and James H.
Godbold. 1999. ``Volume of Ventricular Blood Is an Important Determinant
of Outcome in Supratentorial Intracerebral Hemorrhage.'' \emph{Critical
Care Medicine} 27 (3):617--21.
\url{http://journals.lww.com/ccmjournal/Abstract/1999/03000/Volume_of_ventricular_blood_is_an_important.45.aspx}.

\leavevmode\hypertarget{ref-wang_hematoma_2013}{}%
Wang, Shuo, Min Lou, Tian Liu, Deqi Cui, Xiaomei Chen, and Yi Wang.
2013. ``Hematoma Volume Measurement in Gradient Echo MRI Using
Quantitative Susceptibility Mapping.'' \emph{Stroke} 44 (8):2315--7.
\url{https://doi.org/10.1161/STROKEAHA.113.001638}.

\leavevmode\hypertarget{ref-webb_accuracy_2015}{}%
Webb, Alastair J. S., Natalie L. Ullman, Tim C. Morgan, John Muschelli,
Joshua Kornbluth, Issam A. Awad, Stephen Mayo, et al. 2015. ``Accuracy
of the ABC/2 Score for Intracerebral Hemorrhage Systematic Review and
Analysis of MISTIE, CLEAR-IVH, and CLEAR III.'' \emph{Stroke} 46
(9):2470--6. \url{https://doi.org/10.1161/STROKEAHA.114.007343}.

\end{document}


